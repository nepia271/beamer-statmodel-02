\documentclass[dvipdfmx,cjk]{beamer}

%
% Choose how your presentation looks.
%
% For more themes, color themes and font themes, see:
% http://deic.uab.es/~iblanes/beamer_gallery/index_by_theme.html
%



%%%%%%%%%%%%%%%%%%%%%%%%%%%%%%%%%%%%%%%%%%%%%%%%%% プリアンブル

% ゴシック体
\renewcommand{\kanjifamilydefault}{\gtdefault}

% スライド番号
\setbeamertemplate{footline}[frame number]

% 圏点機能
\usepackage{pxrubrica}

\mode<presentation>
{
  \usetheme{default}      % or try Darmstadt, Madrid, Warsaw, ...
  \usecolortheme{default} % or try albatross, beaver, crane, ...
  \usefonttheme{default}  % or try serif, structurebold, ...
  \setbeamertemplate{navigation symbols}{}
  \setbeamertemplate{caption}[numbered]
} 

\usepackage[english]{babel}
\usepackage[utf8]{inputenc}
\usepackage[T1]{fontenc}

\title{確率モデル入門\\確率の用語整理、確率変数、確率分布}
\author{nepia271}
\institute{Liberal Arts for Tech}
\date{2020/05/30}



%%%%%%%%%%%%%%%%%%%%%%%%%%%%%%%%%%%%%%%%%%%%%%%%%% 本文
\begin{document}


%--------------------------------%
\begin{frame}
  \titlepage
\end{frame}


%--------------------------------%
%\begin{frame}{Outline}
%  \tableofcontents
%\end{frame}


%--------------------------------%
\begin{frame}{講義(セミナー前半)40〜60分}\pause

\begin{block}{目標}
    確率がわかり、確率モデルが身近になること
\end{block}
\pause

\vskip 1cm

\begin{block}{アジェンダ}
    \begin{itemize}
    \item 試行
    \item 標本空間と事象
    \item 確率変数
    \item 確率分布
        \begin{itemize}
        \item 離散確率分布
        \item 連続確率分布
        \end{itemize}
    \end{itemize}
\end{block}

\end{frame}


%--------------------------------%
\begin{frame}{全体の構成}\pause

\begin{itemize}
    \item なぜ「確率」が必要なのか?\pause
    \item 確率の用語整理\pause
    \item 確率変数とは?確率分布とは?\pause
    \item Pythonで例を見る
        \begin{itemize}
        \item 離散確率分布
        \item 連続確率分布
        \item 観測データの分布への収束
        \end{itemize}
\end{itemize}

\end{frame}


%--------------------------------%
\begin{frame}{なぜ「確率」が必要なのか?(1)}\pause

% 端的にいうと「不確かさを定量的に扱いたい」からですが

\begin{itemize}

    \item 「記述統計」ならば「確率」は不要でした
    % たとえば
    \pause
        \begin{itemize}
        \item ヒストグラムを書いてデータを概観する
        \item 平均値、中央値を算出してデータの中心傾向を知る
        \item 分散を算出してデータの散らばり具合を知る
        \end{itemize}

\end{itemize}

\pause
\vskip 1cm
これだけならば、確率の知識は不要!

\end{frame}


%--------------------------------%
\begin{frame}{なぜ「確率」が必要なのか?(2)}

%TODO: ちょっとこのスライド内容悪いのでなんとかしたい

\begin{itemize}

    \item 一方「統計モデル」を使うモチベーションとして……\pause
        \begin{itemize}
        \item データの背後にある "原則" "真理" が知りたい\\\pause
        例)データからわかる、歪んだコインを投げて表が出る%は?\pause
        \item しかも"定量的"に知りたい\\\pause
        ▶(データから)どれくらい歪んでいるのか\\
        ▶(パラメータ推定が)どれくらい信頼できるのか\\
        (「サンプル数が多いほど信頼できる」とは言うけれど?)
        \end{itemize}
        
    \pause
    \item はたまた「統計モデリング」では……
    \pause
        \begin{itemize}
        \item 実世界のデータから筋のよい「確率モデル」を記述したい
        \item そのモデルがどれくらい本当なのかを議論したい
        \end{itemize}

    \pause

\end{itemize}

\vskip 1cm
データがまず与えられる。\\ \pause
その背後にある "原則" "真理" が知りたくてモデルを考える。 \pause

\vskip 1cm
データの背後について定量的に議論に解釈するためには、\\
確率の知識が必須!

\end{frame}


%--------------------------------%
\begin{frame}{まずは確率の用語整理から……}

※資料からかいつまんで紹介\pause

\vskip 1cm

\begin{description}

    \item[> 試行(trial)]\mbox{}\\
            実験や観測などを行うことです。
            確率モデルに基づいて論理展開を行っていくにあたっては、
            試行を行った結果を確率的に解釈していきます。\pause

    \item[> 標本空間(sample space)]\mbox{}\\
            試行の結果を要素とする集合です。\pause

    \item[> 事象(event)]\mbox{}\\
            標本空間の部分集合です。

\end{description}

\end{frame}


%--------------------------------%
\begin{frame}{確率変数とは?}\pause

確率変数とは「ランダムな値をとる」変数です\pause

\vskip 1cm

\begin{itemize}

    \item (注意)確率変数は○○ではありません \pause
        \begin{itemize}
        \item 確率変数は「とる値の集合」ではありません\\ \pause
                それは「標本空間」です \pause
        \item 確率変数は「具体的な実現値」ではありません\\ \pause
                それは「試行」です \pause
        \item 確率変数は「起こりうる事柄」ではありません\\ \pause
                それは「事象」です \pause
        \end{itemize}
        % ここまで聞いて、どう思いました??
\end{itemize}

\vskip 1cm

(いやあ、難しいですね……)

\end{frame}


%--------------------------------%
\begin{frame}{余談:ランダムという概念は実はとっても難しい} \pause

% ランダムの概念はどう発展してきたか
% http://www2.odn.ne.jp/tadaki/miyabe2-lss2012.pdf

\begin{itemize}
    \item 長年、確率のことについては、\\
    すべての事象が等確率で起こるとして、\\
    場合の数で考えられていた(古典的確率) \pause
    \item 確率 = 求める事象/全事象 \pause
    \item しかしこれでは、上手に表現することができないこともあった
    (すべての場合の数を数えるの無理なことがある) \pause

    \item だが、20世紀初頭のこと \\ \pause
    アンドレイ・コルモゴロフというロシアの数学者が、 \\
    確率論の公理化を、みごと完成させた!\\
    (公理的確率論) \pause

    \item ただし、集合論・測度論・ルベーグ積分を駆使して……  \pause
\end{itemize}

\vskip 1cm

(結論)ランダムも、確率変数も、むずかしいので、\\
深く考えないことがオススメです。

% どうしても興味のある人はWikipediaで
% 「確率」の「基礎概念の数学的定義」を見てみましょう
% たぶん、泡吹きます

\end{frame}


%--------------------------------%
\begin{frame}{確率分布とは?} \pause

\begin{itemize}

    \item 「各々の値をとる確率」を表す分布
        \begin{itemize}
        \item 例: コインを投げたとき……
        \item コインが表を取る確率: $50\%$
        \item コインが裏を取る確率: $50\%$  \pause
        \end{itemize}

    \item ヒストグラムをサンプル数で割ったものと「似ています」
        \begin{itemize}
        \item じつのところ「そうではない」のですが
        \item 無限にデータ数があるなら一致していきます
        \end{itemize}

\end{itemize}

\end{frame}


%--------------------------------%
\begin{frame}{離散確率分布・連続確率分布} \pause

\begin{itemize}

    \item 離散確率分布とは? \pause
        \begin{itemize}
        \item 確率変数$X$が\kenten{離散値}をとる場合の確率分布です \pause
        \item 例)コインの表裏、サイコロの出目  \pause
        \item 離散一様分布、二項分布 \pause
        \end{itemize}

    \item 連続確率分布とは? \pause
        \begin{itemize}
        \item 確率変数$X$が\kenten{連続値}をとる場合の確率分布です \pause
        \item 例)花弁の長さ、16歳男子の身長 \pause
        \item 正規分布、ポアソン分布、連続一様分布 \pause
        \end{itemize}

\end{itemize}

\vskip 1cm

離散ならとりうる値ごとに確率が出せますが、\\
連続ではそうはいきません。\\ \pause
「身長が170cmジャスト」とはいえないからです。

\end{frame}


%--------------------------------%
\begin{frame}{よくわからない} \pause

\begin{itemize}
    \item Pythonで動かしてみましょう
    \item 実際に動いている様子で理解のヒントになるかも
\end{itemize}

\end{frame}


%--------------------------------%
\begin{frame}{離散確率分布の例:サイコロ}


\end{frame}


%--------------------------------%
\begin{frame}{連続確率分布の例:正規分布}


\end{frame}


%--------------------------------%
\begin{frame}{サンプルを増やすと真の分布に漸近する}


\end{frame}



%--------------------------------%
\begin{frame}{講義のまとめ} \pause

\begin{itemize}
    \item データが与えられ、背後の法則を考えたい \pause
    \item 法則を考えるために「確率モデル」が利用される \pause
    \item 確率への理解が必要になる(出力を理解することが必要) \pause
    \item しかし、確率の基礎事項は混乱しやすい\\
          確率を一発で理解しようとするのは大変…… \pause
\end{itemize}

\vskip 1cm
ではどうすれば? \pause

\begin{itemize}
    \item ゆっくり丁寧にやることで理解する \pause
    \item Pythonで動かすことで理解する \pause
        \begin{itemize}
        \item 紙とペンでやるよりも
        \item Pythonのほうがカンタンです
        \end{itemize}
\end{itemize} \pause

\vskip 1cm
少し休憩をしたのち、ハンズオンに入ります。

\end{frame}


%--------------------------------%
\begin{frame}{ハンズオン} \pause

\begin{itemize}
    \item sklearn.dataset を読み込む \pause
    \item 適当なヒストグラムを書く \pause
    \item 何の数学的分布に近いか見てみる \pause
\end{itemize}

\vskip 1cm
その他、素朴な疑問について\\確率に限らず拾っていく時間とします

\end{frame}


%===============================%
\end{document}



%%%%%%%%%%%%%%%%%%%%%%%%%%%%%%%%%%%%%%%%%%%%%%%%%% メモ
% https://www.hakodate-ct.ac.jp/~tokai/tokai/doc2013/souzou4.html
% 確率変数と確率分布
% Q1. 確率変数とは何か?
% 難Q2. 普通の変数と確率変数の違いは何か?
% Q3. 連続確率分布とは何か?
% Q4. 離散確率分布とは何か?
% 難Q5. 離散確率分布における確率関数とは何か?
% 難Q6. 連続確率分布における密度関数とは何か?
% 難Q7. 身近にある離散確率分布と連続確率分布の例をそれぞれ挙げよ


